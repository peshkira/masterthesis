\chapter*{Abstract}
\vspace{-1cm}
Information Technology enables us to organise and manage our digital content into collections in an easy fashion. As a result, massive volumes of data are produced each day. However, it creates a huge set of technical and social issues regarding its safety and long-term accessibility.

Digital Preservation copes with issues related to hardware and software obsolescence and tries to keep our digital content accessible in the long term.

In order to make a meaningful decision about the course of action that should be chosen for a digital collection, preservation planning is conducted. 
The result of this rather complex and time-consuming process is a preservation plan.
A preservation plan is an artefact that specifies a concrete action for the preservation of a set of objects and includes potential alternative actions and the reasons for the decision-making. The decision is based on knowledge about the content and the evaluation results of experiments conducted over sample objects of the collection.
Inarguably, a content profile which is a thourough description of the collection and a small set of representative sample objects is crucial for effective planning.

In general, the content profiling process consists of three parts; characterisation, aggregation and analysis.
Characterisation is responsible for the extraction of meta data and the identification of digital objects, while aggregation offers a compressed view on them.
In the last step of analysis, relevant aspects of the content are found and presented for further processing by preservation planning.

Because of the large volume of data, planners face many technical challenges. On the one hand, characterising millions of digital objects is a cumbersome and error prone process. On the other hand, aggregating output of various characterisation tools with complex output schemas is a highly tedious task that requires the expertise of preservation experts and is almost impossible on large scales. The lack of a thorough description and overview of the data often forces planners to select sample objects at random or based on a single property of the data. This results in subsets that are not representative and could lead to biased experiments.

The current state of the art does not offer solutions that are able to automatically create an in-depth profile of a significantly large set of digital objects, select representative samples and expose them in a semi-structured format.	

In this thesis we observe the existing gaps in terms of content profiling and its importance within preservation
planning.
The contribution of this work is a conceptual solution of the content profiling problem, how it could be approached and a software prototype implementing the process.
The presented prototype can operate on collections of about a million objects in scale. It helps to conduct an in-depth analysis, as well as select sample objects based on different algorithms.
We evaluate the prototype using data collections of significant size in two case studies.