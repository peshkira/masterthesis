\section{Content \& Digital Preservation}
Information Technology enables us to write our thoughts and ideas down, to capture moments of our lives in videos and photos, to listen to our favorite songs and so much more in an easy fashion. It completely changed the way of how we think about content. And as it becomes easier and cheaper to create, edit, manipulate, store and share large amounts of digital objects, users often grow unaware of the problems that arise with the digital content they create.

A single sheet of paper put in a normal environment can easiliy endure a number of decades and will still be readable and accessible. A digital object, a file that contains the same content, often doesn't stand a chance of living through the next decade. Hardware failures, software obscolescence, changed environment, lack of backup copies are just a small set of examples of what may occur to your digital objects and render you unable to access your content again.

Digital Preservation is aiming to preserve digital content through the years and make it accessible, readable and understandable for longer periods of time. In the last years growing awareness of digital preservation problems is seen throughout scientific communities, memory institutions and business enterpires. These create solutions and apply different preservation actions on content in order to tackle many problems on different levels.

%explain a bit about preservation planning and make a flow to the motivation
\section{Motivation}
% talk more about preservation planning
% why is it important, the steps in consists of
% automation and its importance.
% concetrate on collection profiling
% why is it important to know what do we have
% in the collections.
\section{Problem Statement}
Very often collection and content profiling is done on a very higher level, that
is often insufficient for a planning process, especially if this process is to be automated.
The current state of the art is collecting simple metrics, such as collection size, and number of elements as well as the formats present in the collection. These measurements are important but are not the only ones needed for the creation of an efficient preservation plan.
In order to create a plan a small subset of the collection is needed in order to conduct some experiments. Based on the results, recommendations and decisions about the preservation actions of the whole collection is needed. Thus the choosing of a representative collection subset is a very important process, that is unfortunately often taken lightly.

\section{Aim Of The Work}

\section{Methodological Approach}

\section{Structure Of The Work}
