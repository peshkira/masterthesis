\section{Content \& Digital Preservation}
The information age enables us to produce, transfer and share massive volumes of data freely in an easy fashion. Scientists all over the world conduct complex research experiments and simulations that produce such an enormous amount of information that was unimaginable just a couple of decades ago. In Microsoft Research's book Jim Gray gives a simple example of the order of magnitude of data volume that is going to be produced; The Large Synoptic Survey Telescope\footnote{http://www.lsst.org/lsst/about} will produce around 1.3 petabytes of information only it its first year of operation, which is more than any other telescope in history has produced \cite{Gray:2009:fourthparadigm}.

The World Wide Web has certainly played an important role as a catalyst of this data growth.
As it becomes easier and cheaper to create, edit, manipulate, store and share large amounts of digital objects, ordinary people often grow unaware of the problems that arise with the digital content they create.

A single sheet of paper put in a normal environment can easily endure a number of decades and will still be readable and accessible. A digital object, a file that contains the same content, often doesn't stand a chance of living through the next decade. Hardware failures, software obsolescence, changed environment, lack of backup copies are just a small set of examples of what may occur to your digital objects and render you unable to access your content again.

Digital Preservation is aiming to preserve digital content through the years and make it findable, accessible, readable and understandable for longer periods of time. In the last years growing awareness of digital preservation problems is seen throughout scientific communities, memory institutions and business enterprises. These create solutions and apply different preservation actions on content in order to tackle many problems on different levels.

%explain a bit about preservation planning and make a flow to the motivation
\section{Motivation}
% talk more about preservation planning
% why is it important, the steps in consists of
% automation and its importance.
% concetrate on collection profiling
% why is it important to know what do we have
% in the collections.
Currently content holding institutions such as Web Archives posses huge amounts of harvested web sites and all linked and related content, such as images, video, style sheet files, etc. Crawling and storing the bits and bytes is only the first step of preserving all this content. Much work has to be done in capturing the significant properties of such dynamic content as the World Wide Web, since the harvested content gets outdated almost immediately after the crawl has been done. This is big issue will definitely be part of future research, however it is not the focus of this thesis.

Regardless of the origin of the content (web archive, scientific data, personal audio and video collections, etc.) preservation planning has to be done in order to be able to identify and apply the most meaningful preservation action and thus ensure the long-term safety and accessibility. And since content is continuously evolving this process has to be repeated on a regular basis.

Before all this however, there are some very important and necessary steps that provide valuable input to planning, which are currently not always done efficiently or not done at all.
For example, the content has to be characterized and all the meta data that is extracted has to be aggregated and analyzed before all potential preservation action alternatives can be found and evaluated. Based on this analysis the content that is to be preserved can be split up into meaningful heterogeneous sub-collections with similar characteristics and significant properties.
 
Furthermore, such an aggregation of the similarities between single parts of the content can enable more efficient stratification of representative elements. These representative subsets 
are the most important input for the experimentation phase during the planning process and based on these and an objective tree defined in another step of the process the most effective preservation action is chosen.
%cite objective tree and representatives.
The current state of the art unfortunately does not specify a concrete and well-defined way of how content profiling should be done and what information it should aggregate. What is more, there are almost no evaluation possibilities of the results that characterization tools offer. Some tools try to give confidence level, which is a first step towards such a validation, however it is not enough. A simple example of such uncertainties is character set encoding of html files. Many files specify a UTF-8 encoding, however the real encoding used in the file is different. Some tools are able to detect this, whereas others just report the declared encoding. This is only one of many examples of such uncertainties that arise due to missing quality assurance steps during the characterization process. Such seemingly minor uncertainties can cause huge impact on the chosen preservation action and thus on the final result of the preserved content.

Therefore, a specific way of aggregating large amounts of meta data and its multidimensional representation would be of great value. Only with such a content profile efficient preservation planning can be achieved.

\section{Problem Statement}
Very often collection and content profiling is done on a very higher level, that
is often insufficient for a planning process, especially if this process is to be automated.
The current state of the art is collecting simple metrics, such as collection size, and number of elements as well as the formats present in the collection. These measurements are important but are not the only ones needed for the creation of an efficient preservation plan.
For example the information about the size of the whole collection is not enough. Other size related measures, such as the overall size of all files in a collection that have a specific format, or the average size of files with a mime type 'text/plain' can be much more significant and helpful in a preservation planning process. These are only a few of the examples of multi dimensional characteristics that play an important role in content profiling.

Furthermore, the creation of a plan requires a small subset of the collection in order to conduct some experiments over it. Based on the results, recommendations and decisions about the preservation actions of the whole collection are produced. Thus the choosing of a representative collection subset is a very important process, that is unfortunately often taken lightly, e.g. done randomly.

Currently, there is no tool that meets these planning requirements.


\section{Aim Of The Work}
In this thesis the author aims to analyze the requirements for such a content profile process and to create a well-defined way generate and represent it, so that it can be used as input to other tools, e.g. preservation planning components.
The architecture of a software tool that is able to read the characteristics of large amount of files and generate a content profile is to be designed as well as a prototype to be implemented. The prototype will make use of some algorithms to retrieve representative subsets based on predefined measures and characteristics.

\section{Methodical Approach}
In a first step a research of the current methods in research projects and institutional practices regarding content profiling will be done. Based on this a short analysis of the existing gap between the idea of content profiling and the actual steps done will be carried out. 

The main part of the thesis will be the creation of a specification and the architecture of a software tool that is able to profile larger amounts of data and produce output conforming to that specification. In the next step research about applicable algorithms able to find a (representative) subset of a given collection of characterized digital objects will be performed and a prototype implementation will be created for the content profiling tool. 

Afterwards a use case study will be conducted, where the produced prototype will be applied on a the Open GovDocs1 collection provided by the Digital Corpora for scientific experiments, which consists of nearly 1 million files and has a total of almost half a terabyte. 

In the last step an evaluation of the tool and the methods applied will be done and based on it a conclusion will be drawn.

%TODO make references to the chapters
%clarify one or the use cases.
\section{Structure Of The Work}
This thesis is structured as follows: The next chapter offers an overview of digital preservation and preservation planning. The state of the art regarding content profiling and some observations of the author about the the current related work. In chapter 3 a theoretical view of content profiling is presented, where its requirements, issues and open challenges are discussed. Chapter 4 presents the architecture and gives deeper insight into a prototype tool that is designed and implemented as part of this master thesis. The following chapter 5 describes a use case that was done in order to evaluate the tool and the conclusions drawn. In the last chapter a summary of this thesis is provided as well as the open issues and next steps regarding content profiling that have to be taken in future research.